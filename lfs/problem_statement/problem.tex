\problemname{LFS}

\begin{center}
\includegraphics[width=12cm]{lfs-lfs.png}
\end{center}

You are designing the new videogame Live Fight Simulator. A level is defined by a string $s$ of length $n$ consisting of lowercase English letters, where each letter represents a type of enemy that appears in sequence.

To analyze the gameplay balance, you need to measure how repetitive different parts of the level are. You will consider $q$ specific contiguous segments $s[l, r]$ of the level, with $1 \le l \le r \le n$.

For each of these queries, you want to compute the length of the LFS (Longest Frequent Substring). Formally, for any string $t$:

\begin{itemize}
\item let $f(t)$ be the maximum frequency of any substring in $t$;
\item let $|\text{LFS}(t)|$ be the maximum length of a substring of $t$ that appears exactly $f(t)$ times.
\end{itemize}

For each query $(l, r)$, you must compute $|\text{LFS}(s[l, r])|$, which represents the maximum length among the most repetitive patterns of enemy spawns in that part of the level.

A string $x$ is a substring of a string $y$ if $x$ can be obtained from $y$ by the deletion of several (possibly, zero or all) characters from the beginning and several (possibly, zero or all) characters from the end.


\section*{Input}
The first line contains two integers $n$, $q$ ($1 \le n \le 5 \cdot 10^5$, $1 \le q \le 5 \cdot 10^5$)~--- the length of the sequence and the number of level segments to analyze.

The second line contains a string $s$ of length $n$ consisting of lowercase English letters.

Each of the next $q$ lines contains two integers $l$, $r$ ($1 \le l \le r \le n$), representing a query on the substring $s[l, r]$.

\section*{Output}
Print $q$ lines. The $i$-th line must contain a single integer: the value of $|\text{LFS}(s[l, r])|$ for the pair $(l, r)$.




\section*{Samples}
In the first example:

\begin{itemize}
\item In the first query, the substring is $t = s[1, 1] = \texttt{"a"}$. The maximum frequency of a substring inside $\texttt{"a"}$ is $1$, reached by $\texttt{"a"}$ itself, whose length is $1$. Therefore, the answer is $1$.
\item In the second query, the substring is $t = s[1, 5] = \texttt{"ababa"}$. The maximum frequency of a substring inside $\texttt{"ababa"}$ is $3$, reached by $\texttt{"a"}$, whose length is $1$. Therefore, the answer is $1$.
\item In the third query, the substring is $t = s[1, 4] = \texttt{"abab"}$. The maximum frequency of a substring inside $\texttt{"abab"}$ is $2$, reached by $\texttt{"a"}$, $\texttt{"b"}$ and $\texttt{"ab"}$. Among these strings, the one with maximum length is $\texttt{"ab"}$, whose length is $2$. Therefore, the answer is $2$.
\item In the fourth query, the substring is $t = s[2, 5] = \texttt{"baba"}$. The maximum frequency of a substring inside $\texttt{"baba"}$ is $2$, reached by $\texttt{"a"}$, $\texttt{"b"}$ and $\texttt{"ba"}$. Among these strings, the one with maximum length is $\texttt{"ba"}$, whose length is $2$. Therefore, the answer is $2$.
\end{itemize}
