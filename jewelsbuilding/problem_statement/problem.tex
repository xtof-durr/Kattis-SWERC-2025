\problemname{Jewels Building}

\begin{center}
\includegraphics[width=10cm]{jewelsbuilding-jewels.png}
\end{center}

You are playing a fantasy game where you start with a row of $n$ power crystals. The $i$-th crystal has energy level $a_i$.

You can perform the following operation any number of times:
\begin{itemize}
\item choose a consecutive group of identical crystals, that is, choose $l$ and $r$ ($1 \leq l \leq r \leq |a|$) such that $a_l = a_{l+1} = \ldots = a_r$;
\item fuse them into a single crystal whose energy becomes $r - l + 1$, obtaining the new sequence $[a_1, \ldots, a_{l-1}, r-l+1, a_{r+1}, \ldots, a_{|a|}]$.
\end{itemize}

Note that you may also choose $l = r$.

You want to craft a specific configuration of crystals with energy levels $b_1, \ldots, b_m$. Determine whether it is possible.


\section*{Input}
Each test contains multiple test cases. The first line contains the number of test cases $t$ ($1 \leq t \leq 500$). The description of the test cases follows.

The first line of each test case contains two integers $n$, $m$ ($1 \leq m \leq n \leq 4000$)~--- the number of crystals in the initial and target configurations, respectively.

The second line of each test case contains $n$ integers $a_1, a_2, \ldots, a_n$ ($1 \leq a_i \leq 10^9$)~--- the energy levels of the initial crystals.

The third line of each test case contains $m$ integers $b_1, b_2, \ldots, b_m$ ($1 \leq b_i \leq 10^9$)~--- the desired energy levels of the target crystals.

It is guaranteed that the sum of $n$ over all test cases does not exceed $4000$.

\section*{Output}
For each test case, output $\texttt{YES}$ if you can transform the initial configuration into the target one, and $\texttt{NO}$ otherwise.

% The judge is case-insensitive (for example, $\texttt{YES}$, $\texttt{Yes}$, $\texttt{yes}$, $\texttt{yEs}$ will all be recognized as positive answers).




\section*{Samples}

In the first test case:
\begin{itemize}
\item the initial configuration is $[2, 4, 4, 2, 3]$;
\item after fusing the two crystals in the subarray $[l, r] = [2, 3]$, the configuration becomes $[2, 2, 2, 3]$;
\item after fusing crystals in the subarray $[l, r] = [1, 3]$, the configuration becomes $[3, 3]$;
\item after fusing crystals in the subarray $[l, r] = [1, 2]$, the configuration becomes $[2] = [b_1]$. So the answer is $\texttt{YES}$.
\end{itemize}

In the second test case, it is not possible to obtain $[4, 4]$ starting from $[2, 4, 4, 2, 3]$, so the answer is $\texttt{NO}$.
